\chapter{Drafts}
\section{Shared Libraries vs Static Libraries}
The difference between shared libraries and static libraries is how they handle their dependencies. When static libraries are compiled, all the code that is needed for the functions in the library is compiled into the library. This means that all the code that is needed for the library to function properly is "copied" to where it is needed. Shared libraries on the other hand has a reference to its dependencies. This will ad an extra cost to the execution since the code need to look up where the code it is supposed to run lies. The advantage of this compared to Static libraries is that the size of the library becomes smaller as we avoid replicates of code.


\section{Vectorization vs Non-vectorization}
\emph{\citep{Vectorization}} explains how Julia is faster at executing devectorized code compared to vectorized code.

\emph{\citep{MoreDotsJuliaBlog}}
Ordinary vectorized code is fast, but not as fast as a hand-written loop (assuming loops are efficiently compiled, as in Julia) because each vectorized operation generates a new temporary array and executes a separate loop, leading to a lot of overhead when multiple vectorized operations are combined.